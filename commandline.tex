\documentclass[10pt]{beamer}

\usetheme{Rochester}

\usepackage[english]{babel}
\usepackage[utf8]{inputenc}
\usepackage[T1]{fontenc}
\usepackage{color}
\usepackage{hyperref}
\usepackage{listings}

% defines a "codeblock" with command
\newcommand{\codeblock}[1]
{
  \colorbox{black}{
    \begin{minipage}{\linewidth}
      \texttt{\textcolor{green}{sls@2405} \textcolor{blue}{cmdline \$} \textcolor{white}{#1}}
    \end{minipage}
  }
}

% 2 argument version, shows output
\newcommand{\codeblockWO}[2]
{
  \colorbox{black}{
    \begin{minipage}{\linewidth}
      \texttt{\textcolor{green}{sls@2405} \textcolor{blue}{cmdline \$} \textcolor{white}{#1}
      \newline
      \textcolor{white}{#2}}
    \end{minipage}
  }
}

\title[Getting Comfortable With the command-line]
{Getting Comfortable With the command-line}

\author[ACM Student Lecture Series]{Daniel~Freedman}

\date[September 14th, 2010]

\begin{document}
\begin{frame}
  \titlepage
\end{frame}

\begin{frame}{Outline}
  \tableofcontents
\end{frame}

\section{What Is The Command-Line?}
\begin{frame}{What Is The Command-Line?}
  \begin{itemize}[<+->]
  \item The command-line is a text based interface to interacting with your computer.
  \item To use the command-line, one enters commands with arguments and hits "Enter" to make the command execute.
  \item More Info on \href{http://en.wikipedia.org/wiki/Command-line\_interface}{\textcolor{blue}{Wikipedia}}
  \end{itemize}
\end{frame}

\section{The Command-line On Your Machine}

\subsection{Linux}
\begin{frame}{Xterm}
\begin{itemize}[<+->]
\item Xterm is best known and most compliant terminal emulator for *nix operating systems.
\item Terminals created by Gnome, KDE, XFCE, etc. usually wrap xterm to provide support for tabs and copy/paste with keyboard combos.
\end{itemize}
\end{frame}

\subsection{Mac}
\begin{frame}{Terminal.app}
\begin{itemize}[<+->]
\item Terminal.app is the default terminal provided by Apple in Mac OS X.
\item It is very fast, but only supports 8 colors unlike xterm's 256 (useful for syntax highlighting).
\item Macs can also use xterm by installing X11 and using the provided X11 xterm.
\end{itemize}
\end{frame}

\subsection{Windows}
\begin{frame}{Cygwin}
\begin{itemize}[<+->]
\item Cygwin emulates a unix-like terminal on Windows, but allows for Windows style file paths and syntax.
\item Cygwin requires tools and applications be compiled inside cygwin for compatibility.
\end{itemize}
\end{frame}

\begin{frame}{SUA}
\begin{itemize}[<+->]
\item SUA stands for Subsystem for Unix Applications.
\item SUA is provided by Microsoft as a pure UNIX application layer for Windows, without any compatibility for filepaths or syntax
\end{itemize}
\end{frame}

\section{Common Command-line Tools}

\subsection{Easy Tools}
\begin{frame}[allowframebreaks]{The Easy And Common Tools}
\begin{itemize}
\item echo: Repeats what was typed back out
\codeblockWO{echo "Hello World"}{Hello World}

\item mv: Moves files from arguments 0 through N-1 to argument N, also used for renaming files
\codeblock{mv samples/file1 hi.txt}

\item cp: Copies a file from argument 1 to argument 2
\codeblock{cp hi.txt hello.txt}

\item rm: Removes files specified
\codeblock{rm hello.txt}

\item mkdir: Makes a new directory
\codeblock{mkdir newdir}

\item ls: Lists files in the current directory
\codeblockWO{ls}{hi.txt\newline~newdir\newline~samples}

\item cat: Prints out the contents of a file
\codeblockWO{cat hi.txt}{Hello There}

\item grep: Lists the lines in the specified files that match the given expression
\codeblockWO{grep "Hello" hi.txt}{Hello There}

\item ssh: Remotely log into another machine
\codeblockWO{ssh dfreedm2@yt.acm.uiuc.edu}{\textcolor{gray}{dfreedm2@yt /home/dfreedm2 \$}}

\item less: Saves output from a command into a scrollable window
\codeblockWO{less samples/bigfile}{This is pdfTeX, Version 3.1415926-1.40.10 (TeX Live 2009/Debian) (format=pdflatex 2010.8.27)  11 SEP 2010 17:49\newline~\colorbox{white}{\textcolor{black}{samples/bigfile lines 1-3/1162 0\%}}}

\item ln: With the '-s' flag, ln can make "symlinks" that provide a pointer to the real location of a file.
\codeblock{ln -s hi.txt pointer}

\item sudo: Run a command as the 'root', the super user.
\codeblockWO{whoami}{sls}
\codeblockWO{sudo whoami}{root}

\item dmesg: Print the system log
\codeblockWO{dmesg}{[71012.760576] Initializing CPU\#1\newline
[71012.760576] CPU: L1 I cache: 32K, L1 D cache: 32K\newline
[71012.760576] CPU: L2 cache: 3072K\newline
[71012.760576] CPU 1/0x1 -> Node 0}

\item head: Read only the first $N$ lines from a file. Default $N=20$. Adjustable with the \texttt{-n} flag.
\codeblockWO{head -n 1}{This is pdfTeX, Version 3.1415926-1.40.10 (TeX Live 2009/Debian) (format=pdflatex 2010.8.27)  11 SEP 2010 17:53}

\item tail: Read only the last $N$ lines from a file. Default $N=20$. Adjustable with the \texttt{-n} flag.
\codeblockWO{tail -n 1}{161 words of extra memory for PDF output out of 10000 (max. 10000000)}

\end{itemize}
\end{frame}

\begin{frame}{Output Redirection}
\begin{itemize}
\item Output Redirection is the ability to move text between applications, directly into files, or directly from files.

\item |: Pipe moves text from the application on the left to the application on the right.
\codeblock{dmesg | less}

\item >: Redirect output from a program into a file.
\codeblock{dmesg > log}

\item $<$: Read a file directly into the input of a program.
\codeblock{tr a-zA-Z $<$ log}
\end{itemize}
\end{frame}

\subsection{Screen}

\subsection{Archivers}

\subsection{Version Control}

\section{Package Managers}

\subsection{Linux}

\subsection{Mac}

\subsection{All Machines}

\section{Programming On The Command-line}
\subsection{vim}
\subsection{emacs}
\subsection{nano}

\section{Tips and Tricks}

\section{QA}
\begin{frame}{QA time}
  \begin{center}
    Questions Anyone?
  \end{center}
\end{frame}

\end{document}
