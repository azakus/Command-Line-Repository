\documentclass[10pt]{beamer}

\usetheme{Rochester}

\usepackage[english]{babel}
\usepackage[utf8]{inputenc}
\usepackage[T1]{fontenc}
\usepackage{color}
\usepackage{hyperref}
\usepackage{listings}

\definecolor{mygray}{rgb}{0.95,0.95,0.95}
\definecolor{mygreen}{rgb}{0.0,0.5,0.0}

\lstset{
       language=sh,
       numbers=left, numberstyle=\tiny, stepnumber=1, numbersep=5pt,
       basicstyle=\color{black}\ttfamily\small,
       commentstyle=\color{mygreen}\ttfamily,
       keywordstyle=\color{blue}\ttfamily,
       stringstyle=\color{red}\ttfamily,
       showstringspaces=false,
       backgroundcolor=\color{mygray},
       breaklines,
       frame=single,
}

% defines a "codeblock" with command
\newcommand{\codeblock}[1]
{
  \colorbox{black}{
    \begin{minipage}{\linewidth}
      \texttt{\textcolor{green}{sls@2405} \textcolor{blue}{cmdline \$} \textcolor{white}{#1}}
    \end{minipage}
  }
}

% 2 argument version, shows output
\newcommand{\codeblockWO}[2]
{
  \colorbox{black}{
    \begin{minipage}{\linewidth}
      \texttt{\textcolor{green}{sls@2405} \textcolor{blue}{cmdline \$} \textcolor{white}{#1}
      \newline
      \textcolor{white}{#2}}
    \end{minipage}
  }
}

\AtBeginSection[]{\begin{frame}[shrink]{Outline}\tableofcontents[currentsection]\end{frame}}

\title[Getting Comfortable With the command-line]
{Getting Comfortable With the command-line}

\author[ACM Student Lecture Series]{Daniel~Freedman}

\date[September 14th, 2010]

\begin{document}
\begin{frame}
  \titlepage
\end{frame}

\begin{frame}[shrink]{Outline}
  \tableofcontents
\end{frame}

\section{What Is The Command-Line?}
\begin{frame}{What Is The Command-Line?}
  \begin{itemize}[<+->]
  \item The command-line is a text based interface to interacting with your computer.
  \item To use the command-line, one enters commands with arguments and hits "Enter" to make the command execute.
  \item More Info on \href{http://en.wikipedia.org/wiki/Command-line\_interface}{\textcolor{blue}{Wikipedia}}
  \item At this time, go ahead and download the bundle to follow along: \url{http://www.acm.uiuc.edu/~dfreedm2/cmdline.tar.gz}
  \end{itemize}
\end{frame}

\section{The Command-line On Your Machine}

\subsection{Linux}
\begin{frame}{Xterm}
\begin{itemize}[<+->]
\item Xterm is best known and most compliant terminal emulator for *nix operating systems.
\item Terminals created by Gnome, KDE, XFCE, etc. usually wrap xterm to provide support for tabs and copy/paste with keyboard combos.
\end{itemize}
\end{frame}

\subsection{Mac}
\begin{frame}{Terminal.app}
\begin{itemize}[<+->]
\item Terminal.app is the default terminal provided by Apple in Mac OS X.
\item It is very fast, but only supports 8 colors unlike xterm's 256 (useful for syntax highlighting).
\item Macs can also use xterm by installing X11 and using the provided X11 xterm.
\end{itemize}
\end{frame}

\subsection{Windows}
\begin{frame}{Cygwin}
\begin{itemize}[<+->]
\item Cygwin emulates a unix-like terminal on Windows, but allows for Windows style file paths and syntax.
\item Cygwin requires tools and applications be compiled inside cygwin for compatibility.
\end{itemize}
\end{frame}

\begin{frame}{SUA}
\begin{itemize}[<+->]
\item SUA stands for Subsystem for Unix Applications.
\item SUA is provided by Microsoft as a pure UNIX application layer for Windows, without any compatibility for filepaths or syntax
\end{itemize}
\end{frame}

\section{Common Command-line Tools}

\subsection{Easy Tools}
\begin{frame}[allowframebreaks]{The Easy And Common Tools}
\begin{itemize}
\item echo: Repeats what was typed back out
\codeblockWO{echo "Hello World"}{Hello World}

\item mv: Moves files from arguments 0 through N-1 to argument N, also used for renaming files
\codeblock{mv samples/file1 hi.txt}

\item cp: Copies a file from argument 1 to argument 2
\codeblock{cp hi.txt hello.txt}

\item rm: Removes files specified
\codeblock{rm hello.txt}

\item mkdir: Makes a new directory
\codeblock{mkdir newdir}

\item ls: Lists files in the current directory
\codeblockWO{ls}{hi.txt\newline~newdir\newline~samples}

\item cat: Prints out the contents of a file
\codeblockWO{cat hi.txt}{Hello There}

\item grep: Lists the lines in the specified files that match the given expression
\codeblockWO{grep "Hello" hi.txt}{Hello There}

\item ssh: Remotely log into another machine
\codeblockWO{ssh dfreedm2@yt.acm.uiuc.edu}{\textcolor{gray}{dfreedm2@yt /home/dfreedm2 \$}}

\item less: Saves output from a command into a scrollable window
\codeblockWO{less samples/bigfile}{This is pdfTeX, Version 3.1415926-1.40.10 (TeX Live 2009/Debian) (format=pdflatex 2010.8.27)  11 SEP 2010 17:49\newline~\colorbox{white}{\textcolor{black}{samples/bigfile lines 1-3/1162 0\%}}}

\item ln: With the '-s' flag, ln can make "symlinks" that provide a pointer to the real location of a file.
\codeblock{ln -s hi.txt pointer}

\item sudo: Run a command as the 'root', the super user.
\codeblockWO{whoami}{sls}
\codeblockWO{sudo whoami}{root}

\item dmesg: Print the system log
\codeblockWO{dmesg}{[71012.760576] Initializing CPU\#1\newline
[71012.760576] CPU: L1 I cache: 32K, L1 D cache: 32K\newline
[71012.760576] CPU: L2 cache: 3072K\newline
[71012.760576] CPU 1/0x1 -> Node 0}

\item head: Read only the first $N$ lines from a file. Default $N=20$. Adjustable with the \texttt{-n} flag.
\codeblockWO{head -n 1}{This is pdfTeX, Version 3.1415926-1.40.10 (TeX Live 2009/Debian) (format=pdflatex 2010.8.27)  11 SEP 2010 17:53}

\item tail: Read only the last $N$ lines from a file. Default $N=20$. Adjustable with the \texttt{-n} flag.
\codeblockWO{tail -n 1}{161 words of extra memory for PDF output out of 10000 (max. 10000000)}

\end{itemize}
\end{frame}

\begin{frame}{Output Redirection}
\begin{itemize}
\item Output Redirection is the ability to move text between applications, directly into files, or directly from files.

\item |: Pipe moves text from the application on the left to the application on the right.
\codeblock{dmesg | grep -i usb | less}

\item >: Redirect output from a program into a file.
\codeblock{dmesg > log}

\item $<$: Read a file directly into the input of a program.
\codeblock{tr a-z A-Z $\texttt{<}$ log}
\end{itemize}
\end{frame}

\subsection{Archivers}
\begin{frame}{Tar}
\begin{itemize}
\item Tar is the default compression application in most Unix based OSes.
\pause

\item Tar is quite complicated, so for the purposes of this talk, we will only discuss extracting and creating archives.
\pause

\item Tar just concatenates files into one blob, so other helper programs such as gzip and bzip2 are used to compress the blob.
\pause

\item Creating a tar archive can be done with "tar czvf", which invokes gzip as the compression program
\codeblockWO{tar czvf tarball.tar.gz hi.txt}{hi.txt}

\item Extracting an archive can be done with "tar xvf", which automatically picks the compression program to use for decompression \textit{on most machines}.
\codeblockWO{tar xvf tarball.tar.gz}{hi.txt}
\end{itemize}
\end{frame}

\section{Package Managers}

\subsection{Linux}
\begin{frame}{dpkg}
\begin{itemize}[<+->]
\item dpkg is the package management system for Debian based systems.
\item dpkg is rather barebones, so a good frontend for this system is found in aptitude.
\item aptitude handles all the package fetching and dependency resolution for you, in a safe and easy way.
\item \texttt{\color{blue}aptitude search} returns packages that match what you search for.
\item \texttt{\color{blue}aptitude install} will install the package, and install all dependencies.
\item \texttt{\color{blue}aptitude update} will pull update the package listings from the server, and alert you to new updates.
\item \texttt{\color{blue}aptitude safe-upgrade} will try to safely upgrade all the packages.
\item Other uses exist, but read the man page.
\end{itemize}
\end{frame}

\begin{frame}{rpm}
\begin{itemize}[<+->]
\item rpm is the package management system for Red Hat based systems.
\item yum is the most used frontend for these systems.
\item \texttt{\color{blue}yum search -C} returns packages that match what you search for.
\item \texttt{\color{blue}yum install} will install the package, and install all dependencies.
\item \texttt{\color{blue}yum update} will pull update the package listings from the server, and alert you to new updates.
\item \texttt{\color{blue}yum upgrade} will try to safely upgrade all the packages.
\end{itemize}
\end{frame}

\subsection{Mac}
\begin{frame}{brew}
\begin{itemize}[<+->]
\item Macs do not have a built in package management system as powerful as dpkg, but there are good third party ones.
\item The most popular/usable is brew.
\item brew is based on git, which means that you must install git first using the Mac installers
\item Just follow the \href{http://help.github.com/mac-git-installation/}{\color{blue}Guide to Installing Git}.
\item \texttt{\color{blue}brew search} returns packages that match your search terms.
\item \texttt{\color{blue}brew install} fetchs, compiles, and installs a package.
\item \texttt{\color{blue}brew outdated} returns which packages are outdated.
\item \texttt{\color{blue}brew install \$(brew outdated)} upgrades outdated packages.
\end{itemize}
\end{frame}

\subsection{All Machines}
\begin{frame}{Other}
\begin{itemize}[<+->]
\item If a package you need is not included in brew or your linux repo's repository, you're going to have to build it yourself.
\item Fortunately, most open source software comes with a standard set of tools to help build the project and install it.
\item This technique does not handle dependencies, so you'll have to manage those on your own.
\item \texttt{\color{blue}configure} is a script that automatically checks if you have the required libraries installed and sets up the Makefile.
\item Assuming all went well, \texttt{\color{blue}make} will read the Makefile and compile the code.
\item \texttt{\color{blue}make install} will then install the code into the right spot.
\end{itemize}
\end{frame}

\section{Programming On The Command-line}

\subsection{Editors}
\begin{frame}{Editors}
\begin{itemize}[<+->]
\item Commandline editors are loved by some, confusing to most, and the topic of many flamewars.
\item The two big command line editors are \texttt{\color{blue}vim} and \texttt{\color{blue}emacs}, which are very different and have whole communities behind them.
\item Another smaller editor is \texttt{\color{blue}nano}, which is less powerful, but more accessible.
\item These two editors are so vast that they quickly fall out of the scope of this talk, but you can learn more about them from the links below.
\item Vim: \url{http://www.vim.org}.
\item Emacs: \url{http://www.gnu.org/software/emacs/}.
\item Nano: \url{http://www.nano-editor.org/}.
\end{itemize}
\end{frame}

\subsection{screen}
\begin{frame}{screen}
\begin{itemize}[<+->]
\item screen is a terminal multiplexer, and allows you to have multiple running terminal sessions at once and in the background.
\item In addition, screen can be used to run commands remotely that can survive terminal disconnects.
\item screen has many features outside the scope of this talk.
\item More info can be found on the homepage: \url{http://www.gnu.org/software/screen/}.
\end{itemize}
\end{frame}

\subsection{Version Control}
\begin{frame}{Version Control}
\begin{itemize}[<+->]
\item Version Control systems are vital to correctly maintaining code which multiple people access.
\item The two major players are \texttt{\color{blue}git} and \texttt{\color{blue}svn}.
\item The commands for both are near identical, and you'll have \texttt{\color{blue}svn} experience in classes, so we will only cover \texttt{\color{blue}git} here.
\item \texttt{\color{blue}git clone git://github.com/azakus/Command-Line-Repository.git} will clone the repository that contains this talk.
\item Now create a file with your name in it called name.txt.
\item \texttt{\color{blue}git add name.txt} will tell git to add the file's changes to the repo.
\item \texttt{\color{blue}git commit} will then commit the changes to the repo.
\item \texttt{\color{blue}git log} will show you that you commited the file to the repo!
\item \href{http://www.sourcemage.org/Git\_Guide}{\color{blue}SourceMage} has a rather useful guide to beginning, intermediate, and advanced \texttt{git} useage.
\end{itemize}
\end{frame}

\subsection{Man Pages}
\begin{frame}{man}
\begin{itemize}[<+->]
\item Man pages are the manuals installed with most applications on the command-line.
\item To read a man page, just type \texttt{\color{blue}man} followed by the program name to launch it's manual.
\item Man pages exist for everything from trivial programs to language functions and libraries.
\end{itemize}
\end{frame}

\subsection{Shell}
\begin{frame}{Shell}
\begin{itemize}[<+->]
\item Shell is a programming language purpose built to patch together the output from applications.
\item It is a very lightweight language with only a few keywords.
\item \texttt{\color{blue}man sh} should give you all the syntax.
\item Commands wrapped with \texttt{\$(COMMAND)} return the output of that command.
\item How would you get the total size of a directory? \tiny{wc -c gives size in bytes}.
\end{itemize}
\end{frame}

\begin{frame}{Example Program}
\lstinputlisting{cmdline/samples/sample.sh}
This is also known as \texttt{\color{blue}du -s}.
\end{frame}

\section{Tips and Tricks}
\begin{frame}{Tips and Tricks}
\begin{itemize}
\item xclip can put text into and out of the copy/paste buffer.
\item Ctrl-a goes to front of line, Ctrl-e goes to end of line.
\item Ctrl-u clears a whole line, Ctrl-k clears a line from the cursor onwards.
\item Ctrl-w removes a word from the line.
\item Invoking a command with a \& afterwards will run it in the background.
\item Ctrl-z pauses a program, \texttt{\color{blue}fg} puts it back in the foreground, \texttt{\color{blue}bg} runs it in the background.
\item \texttt{\color{blue}alias} allows you to set up macros in the terminal.
\end{itemize}
\end{frame}

\section{QA}
\begin{frame}{QA time}
  \begin{center}
    Questions Anyone?
  \end{center}
\end{frame}

\end{document}
